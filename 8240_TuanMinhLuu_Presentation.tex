\documentclass{beamer}
\usetheme{Boadilla}

\title{Reproducing FastText Sentiment Analysis}
\subtitle{Using VADER Sentiment }
\author{Tuan Minh Luu}
\institute{Macquarie University}
\date{\today}

\usepackage{listings}  % For code listings
\usepackage{hyperref}  % For hyperlinks
\usepackage{graphicx}  % For including images
\usepackage{color}
\usepackage{amsmath}   % For mathematical symbols

% Customize listings
\lstset{
    basicstyle=\ttfamily\small,  % Typewriter font for code
    breaklines=true,              % Wrap long lines
    keywordstyle=\color{blue},    % Style for keywords
    commentstyle=\color{green},   % Style for comments
    stringstyle=\color{red},      % Style for strings
    frame=single,                 % Add frame around the code
    showstringspaces=false,       % Do not show spaces in strings
}

\begin{document}

% Title slide
\begin{frame}
    \titlepage
\end{frame}

% Table of contents
\begin{frame}
    \frametitle{Outline}
    \tableofcontents
\end{frame}

\section{Introduction}
\begin{frame}{Introduction}
    In this presentation, we describe the sentiment analysis project using FastText. The tasks involved include:
    \begin{itemize}
                \item Preprocessing and cleaning the BBC news dataset.
    \item Applying VADER sentiment analysis.
    \item Training a FastText model.
    \item Improving model performance with hyperparameter tuning.
    \item Balancing the dataset and re-evaluating the model.
    \end{itemize}
    
\end{frame}

\section{Replication of Original Work}
\begin{frame}
    \frametitle{Replication of Original Work}
   In this project, we replicated the original work on sentiment analysis  using the fastText library developed by Facebook AI Re-search team.
\end{frame}


\section{Data Pre-processing}
\begin{frame}{Data Preprocessing}
The preprocessing steps applied to the dataset were as follows:
\begin{itemize}
\item Lowercasing and removing special characters.
        \item Removing stop words using the NLTK library.
\item Lemmatizing text to its root form.

\end{itemize}
\vspace{0.5cm}
The cleaned data columns include:
\begin{itemize}
    \item Title cleaned.
    \item Description cleaned.
    \item Combined sentiment (using VADER analysis).
\end{itemize}

\end{frame}

\section{Model Training and Evaluation}
\begin{frame}[fragile]{Model Training and Evaluation}
We utilized \texttt{FastText} for the training and evaluation process:
\begin{itemize}
    \item\textbf{Training Parameters:}
\begin{lstlisting}[language=python]
model = fasttext.train_supervised(
        input ='bbc_news_train_combined.txt',       l.r=1.0, epoch=25,wordNgram=2,                  minCount=1, verbose=2)
\end{lstlisting}
\item \textbf{Evaluation Result:}
        \begin{itemize}
            \item Precision: 73.68\%
            \item Racall: 73.68\%
            \item Confusion Matrix: Highlighted True Positive and False Negative.
        \end{itemize}
\end{itemize}
\end{frame}

\section{Hyperparameter Tuning}
\begin{frame}{Hyperparameter Tuning}
A grid search approach was employed for hyperparameter tuning:
\begin{itemize}
    \item Learning rates: 0.01, 0.05, 0.1
    \item Epochs: 25, 50
    \item WordNgrams: 1, 2
    \item Embedding Dimensions: 50, 100
\end{itemize}
\vspace{0.5cm}
The best model achieved a precision of 0.733 using:
\begin{itemize}
    \item Learning rates: 0.01
    \item Epochs: 50
    \item WordNgrams: 2
    \item Embedding dimension: 100
\end{itemize}
The result of the best model is lower than the official model.
\end{frame}

\section{Dataset Balancing}
\begin{frame}{Dataset Balancing}
Oversampling techniques were applied to balance the neutral class, avoid bias in model training. Class distribution after oversampling:
\begin{itemize}
    \item Negative: 17801
    \item Positive: 14401
    \item Neutral: 14401
\end{itemize}
Model trained on the balanced dataset achieved a precision of 98.1\%.
\end{frame}

\section{Comparison of Results}
\begin{frame}[fragile]
    \frametitle{Comparison of Model Results}
    We compared the results of our trained model with the official fastText models using **precision** as the evaluation metric:

    \begin{lstlisting}[language=python]
    test\_data['balanced\_predictions'] = test\_data['text\_only'].apply(lambda x: balanced\_model.predict(x)[0][0])
    \end{lstlisting}

    \begin{itemize}
        \item \textbf{Our Model:} Precision = 98.1\%
        \item \textbf{Official Model}: Precision = 73.68\%
        \item \textbf{Official Model with hyperparameter:} Precision = 73.39\%
    \end{itemize}
    
    \textbf{Precision} was chosen as it measures the proportion of correctly identified languages out of the total predicted. It is a crucial metric when accuracy alone doesn't capture performance in multi-class classification.
\end{frame}

\section{Model Utilities}
\begin{frame}{Model Utilities}
    The FastText-based sentiment analysis model offers several key utilities:
    \begin{itemize}
        \item \textbf{Efficient Sentiment Analysis:} Automates the classification of news articles, providing quick and scalable sentiment analysis for large datasets.
        \item \textbf{Real-Time Classification:} FastText allows for real-time sentiment detection, which is crucial for media monitoring platforms and sentiment-based recommendations.
        \item \textbf{Balanced Dataset Approach:} By employing oversampling, the model handles underrepresented classes effectively, ensuring reliable results across sentiment categories.
        \item \textbf{Generalizability Across Topics:} Demonstrates effective sentiment classification across diverse news topics, useful for media analysts and researchers.
        \item \textbf{Practical Applications for Media Organizations:} Helps automate content tagging, monitor sentiment trends, and analyze public opinion in real-time.
    \end{itemize}
\end{frame}

\section{Model Limitations and Future Improvements}
\begin{frame}{Model Limitations and Future Improvements}
    Despite achieving high accuracy, the model exhibits several limitations:
    \begin{itemize}
        \item \textbf{Overfitting Risk:} High accuracy on the test set could indicate overfitting, especially when the training and testing datasets are closely related.
        \item \textbf{Imbalance in Sentiment Classes:} Although oversampling was employed, the model may still face challenges in identifying minority classes like neutral sentiments.
        \item \textbf{Handling Ambiguous Sentiments:} The model struggles with articles that have mixed or subtle sentiments, leading to misclassification in cases where sentiments are not clear-cut.
        \item \textbf{VADER Limitations:} Using VADER as the sentiment analysis base might be insufficient for complex language structures, leading to inaccuracies in cases involving nuanced language.
    \end{itemize}
\end{frame}

\section{Incorporating Techniques and Tools}
\begin{frame}
    \frametitle{Incorporating Techniques and Tools}
    This project effectively utilized various techniques and tools to accomplish sentiment analysis on the BBC news dataset:

    \begin{itemize}
        \item \textbf{FastText for Text Classification:} Leveraged the FastText library for efficient model training and hyperparameter tuning to perform text classification tasks.
        \item \textbf{Data Cleaning and Preprocessing:} Employed regular expressions, NLTK’s stopwords, and VADER sentiment analysis to clean and preprocess text data effectively.
        \item \textbf{LaTeX for Presentation:} The entire presentation, including code listings, was written using LaTeX.
        \item \textbf{Google Colab for Execution:} Leveraged Google Colab’s cloud-based environment for running and testing large-scale experiments conveniently.
    \end{itemize}
\end{frame}
\section{Conclusion}
\begin{frame}
    \frametitle{Conclusion}
    \textbf{Key Takeaways:}
    \begin{itemize}
\begin{itemize}
    \item \textbf{Improved Sentiment Classification:} Achieved high precision through effective preprocessing and FastText training.
    \item \textbf{Class Imbalance Addressed:} Applied oversampling to enhance model accuracy, particularly for neutral sentiments.
    \item \textbf{Thorough Data Cleaning:} Used stopword removal, lemmatization, and filtering to refine sentiment predictions.
\end{itemize}
    \end{itemize}

    \vspace{0.4cm}
    \textbf{GitHub Repository:}
    The repository contains:
    \begin{itemize}
        \item Python scripts for data cleaning and model training, training on balanced datasets and tuning hyperparameters.
        \item Instructions on reproducing results and applying improvements.
    \end{itemize}
\end{frame}
\end{document}